\documentclass[12pt,letterpaper,titlepage]{article}
\usepackage[spanish]{babel}
\usepackage[utf8x]{inputenc}
\usepackage{amsmath,amssymb}
\usepackage{graphicx}
\usepackage{amsthm}
\usepackage{latexsym}
\usepackage{cancel}
\usepackage{mathrsfs,amsfonts,mathptmx}

% diseño de página
\setlength{\parindent}{5ex}				% sangría
\usepackage[inner=1.6in,outer=1in,top=1in,bottom=1in]{geometry}
\usepackage{setspace}					% interlineado

\usepackage{color,longtable,pdfpages}
\usepackage{graphicx,hyperref}
\usepackage{url,breakurl}
\usepackage{skak}
\setcounter{secnumdepth}{3}
\setcounter{tocdepth}{4} 
\setlength\LTleft{0pt} \setlength\LTright{0pt} % parámetros para tablas largas

%\usepackage[text={5.8in,8.6in},centering]{geometry}
\renewcommand{\spanishoperators}{sen spec d}
\renewcommand{\baselinestretch}{1.6}
\makeatletter \decimalpoint
  \def\th@exercise{%
    \normalfont % body font
    \thm@headpunct{:}%
  }

\usepackage{apacite}
\makeatother
\usepackage{hyperref}
\hypersetup{% bookmarksnumbered,
  bookmarksopen,
  pdfpagelayout=OneColumn,
  pdfview=FitH,
  pdfstartview=FitH,
  pdfborder={0 0 0}}




% \theoremstyle{definition}
\renewcommand{\spanishrefname}{Bibliografía preliminar}

\title{Valuación Actuarial del Plan de Pensiones de los Trabajadores al Servicio del Instituto Guatemalteco de Seguridad Social (IGSS)}
\begin{document}
\begin{titlepage}
	\renewcommand{\thepage}{}
	\pagestyle{empty}
	\maketitle
\end{titlepage}\newpage
\setcounter{page}{2}
\tableofcontents
%%%%%%%%%%%%%%%%%%%%%%%%%%%%%%%%%%%%%%%%%%%%%%%%%%%%%%%%%%%%%%%%%%%%%%%%%%%%%%%%%%%%%%%%%%%%%%%%%%%%%%%%%%%%%%%%%%%%%%%%%%%%	
\newpage
\nocite{*}
\section{Introducción}

La valuación actuarial del Plan de Pensiones de los Trabajadores del Instituto Guatemalteco de Seguridad Social se hará mediante el cálculo de su Reserva Matemática, la cual es aconsejada por ser este de carácter privado.

Se comenzará haciendo un recuento de las prestaciones que este plan otorga a sus asegurados, así como también indicando los requisitos de cada uno.

(...)

\section{Generalidades}

El sistema de financiamiento del Plan de Pensiones de los Trabajadores del Instituto Guatemalteco de Seguridad Social (IGSS) está conformado por el sistema de prima escalonada. Dicho plan posee una normativa de revisión actuarial anual con un período de equilibrio no menor a los 5 años. 

La finalidad del mismo consiste en otorgar como patrón, beneficios complementarios al programa de Invalidez, Vejez y Sobrevivencia (IVS) del Instituto como velador de la seguridad social. El Plan es de carácter obligatorio para todos los trabajadores, contributivo por ambas partes (trabajador y el Instituto), complementario (como anteriormente se expuso) e independiente de los fondos y el manejo de cualquier otro programa que sea parte del Instituto.

Este tipo de Plan, se encuentra entre los catalogados \textit{plan de pensiones ocupacional}. Posee un régimen de contribución definida. Los beneficios que el plan otorga para sus adeptos son: \textit{benificio en caso invalidez, benificio en vejez, beneficio en caso de sobrevivencia, asignación única} y \textit{cuota mortuoria}.

\section{Beneficios}

A continuación se enlistarán a detalle cada uno de los beneficios que el Instituto otorga a sus trabajadores mediante el Plan:

\subsection{Invalidez}

Según el IGSS un asegurado, se considera en estado de \textit{Invalidez} cuando presente una incapacidad para procurarse ingresos económicos como asalariado, en las condiciones en que los obtenía antes de la ocurrencia del riesgo que la originó. 

Entre los requisitos del Plan para que un asegurado sea acreedor del beneficio de invalidez se encuentran:
\begin{enumerate}
	\item Ser declarado inválido por el IGSS
	\item Haber contribuido 24 meses al Plan como mínimo en los ultimos cuatro años inmediatamente anteriores al primer día de invalidez
\end{enumerate}

En donde se le otorgará lo siguiente:
\begin{enumerate}
	\item Una pensión, cuyo monto será un mínimo del 60\% del último salario mensual o al porcentaje que corresponde según tabla del artículo 10 del Acuerdo de Junta Directiva 1135 Modificado por el Acuerdo de Junta Directiva 1362 
	\item La pensión de invalidez parcial se transforma en pensión de invalidez total al cumplir 55 años, mayor de 55 años se le dará pensión de invalidez total
\end{enumerate}

Haciendo énfasis en el monto de la pensión cuando sea pensión por invalidez parcial, la cual es la mitad de la total, gran invalidez es igual al 80\% o el correspondiente según tabla mencionada.

\subsection{Vejez}

Según los requirimientos del Plan, un asegurado es catalogado en estado de Vejez cuando el mismo tenga una edad de 55 años.

Los requisitos del Plan para la pensión de vejez son:
\begin{enumerate}
	\item Haber contribuido 240 meses al Plan
	\item Terminar su relación laboral con el Instituto
\end{enumerate}

Siendo el beneficio de este: una pensión cuyo monto es conforme la tabla del artículo 10 del Acuerdo de Junta Directiva 1135 Modificado por el Acuerdo de Junta Directiva 1362.

\subsection{Sobrevivencia}




























	\end{document}