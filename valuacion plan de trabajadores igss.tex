\documentclass[12pt,letterpaper,titlepage]{article}
\usepackage[spanish]{babel}
\usepackage[utf8x]{inputenc}
\usepackage{amsmath,amssymb}
\usepackage{graphicx}
\usepackage{amsthm}
\usepackage{latexsym}
%\usepackage{mathcal}
\usepackage{cancel}
\usepackage{mathrsfs,amsfonts,mathptmx}
\usepackage{multirow}
% diseño de página
\setlength{\parindent}{3ex}				% sangría
\usepackage[inner=1.6in,outer=1in,top=1in,bottom=1in]{geometry}
\usepackage{setspace}					% interlineado
\usepackage{color,longtable,pdfpages}
\usepackage{graphicx,hyperref}
\usepackage{url,breakurl}
\usepackage{skak}
\usepackage{array}
\setcounter{secnumdepth}{3}
\setcounter{tocdepth}{4} 
\setlength\LTleft{0pt} \setlength\LTright{0pt} % parámetros para tablas largas

%\usepackage[text={5.8in,8.6in},centering]{geometry}
\renewcommand{\spanishoperators}{sen spec d}
\renewcommand{\baselinestretch}{1.6}
\makeatletter \decimalpoint
  \def\th@exercise{%
    \normalfont % body font
    \thm@headpunct{:}%
  }

\usepackage{apacite}
\makeatother
\usepackage{hyperref}
\hypersetup{% bookmarksnumbered,
  bookmarksopen,
  pdfpagelayout=OneColumn,
  pdfview=FitH,
  pdfstartview=FitH,
  pdfborder={0 0 0}}




% \theoremstyle{definition}
\renewcommand{\spanishrefname}{Bibliografía preliminar}

\title{ Planteamiento Teórico del Análisis Actuarial al Plan de Pensiones de los Trabajadores al Servicio del Instituto Guatemalteco de Seguridad Social (IGSS) }
\author{Proyecto de Práctica Profesional Realizado por:\\
	Emilene Analí Romero Marroquín\\
	Universidad de San Carlos de Guatemala\\
	Asesorado por: Dr. Roberto Molina Cruz\\}

\begin{document}
	{\onehalfspacing
\begin{titlepage}
	\renewcommand{\thepage}{}
	\pagestyle{empty}
	\maketitle
\end{titlepage}\newpage
\setcounter{page}{2}
\tableofcontents
%%%%%%%%%%%%%%%%%%%%%%%%%%%%%%%%%%%%%%%%%%%%%%%%%%%%%%%%%%%%%%%%%%%%%%%%%%%%%%%%%%%%%%%%%%%%%%%%%%%%%%%%%%%%%%%%%%%%%%%%%%%%	
\newpage
\nocite{*}
\section{Introducción}

El Plan de Pensiones de los Trabajadores del Instituto Guatemalteco de Seguridad Social, se basa en un método de  financiamiento\footnote{Preferentemente un método de financiamiento de plan de pensiones de carácter social} conocido como Prima Escalonada (PE)\footnote{Que se supone provee de una reserva no decreciente, por lo que en teoría se recurre únicamente a los ingresos obtenidos de las inversiones de la reserva, pero no la reserva en sí misma.}. La situación financiera de un plan financiado con este método, es determinada mediante la Longitud del Período de Equilibrio (LPE).\bigskip

Las últimas valuaciones actuariales realizadas en el Departamento Actuarial y Estadístico, indican que la situación financiera del Plan es inestable, por lo que se quiere estudiar la posibilidad de cambiar el método de financiamiento de Prima Escalonada (PE) al método de Prima Media General (PMG)\footnote{El cual consiste de una contribución constante por un tiempo indefinido.}, siendo éste el más recomendable. La situación financiera de un plan financiado por medio del método de Prima Media General es determinada, por el indicador de su Reserva Matemática (RM). Este documento busca describir dicho método y el indicador RM y compararlos con el método de Prima Escalonada (PE) y el indicador de Longitud de Período de Equilibrio (LPE) ya que el principal objetivo de una valuación periódica de un esquema en curso es examinar la solvencia a largo plazo, es decir, evaluar si bajo el actual método de financiamiento, existe la solvencia para el pago de las prestaciones presentes y futuras y si los fondos de la reserva pueden mantener los niveles requeridos. \bigskip

%Sin embargo en las valuaciones recientes se ha reflejado un "decrecion" de la reserva por lo que se ha visto comprometido el balance de dicho plan, surgiendo de esta situación la necesidad de evaluar la situación financiera desde otra perspectiva para analizar las medidas necesarias así como las recomendaciones (...) para contrarrestar (...) prosiguiendo con una valuación actuarial del plan.\bigskip

%La valuación actuarial del Plan de Pensiones de los Trabajadores del Instituto Guatemalteco de Seguridad Social se hará mediante el cálculo de su Reserva Matemática, la cual es aconsejada por ser este de carácter privado.\bigskip

Por lo tanto se realizará un recuento de las prestaciones, como también indicando los requisitos de cada una, que el plan otorga a sus asegurados. Prosiguiendo con el cálculo de los valores presentes de cada prestación del plan para el cálculo final de la Reserva Matemática total. \bigskip

%%%%%%%%%%%%%%%%%%%%%%%%%%%%%%%%%%%%%%%%%%%%%%%%%%%%%%%%%%%%%%%%%%%%%%%%%%%%%%%%%%%%%%%%%%%%%%%%%%%%%%%%%%%%%%%%%%%%%%%%%%%%%%%%%%%%%%%%%%%%%%%%
\newpage
\section{Antecedentes}

\subsection{Antecedentes de la Seguridad Social en el Mundo}

Durante el siglo XIX con el surgimiento de la Revolución Industrial, en los países conocidos como desarrollados, en Europa o Estados Unidos por ejemplo,  se promovió la creación de una institución que velara por la seguridad económica del trabajador, en caso de este encontrarse en una situación desventajosa para poder proveerse de un modo para obtener ingresos, siendo esta la \textbf{seguridad social}. \\

Fue en Alemania, durante el gobierno del canciller Otto Von Bismarck que se instauró una política social cuyo fin fue eliminar la incertidumbre e inseguridad de los trabajadores. Por lo que para el año 1881 el gobierno de Alemania fijó un programa en materia de política social, a partir del cual los trabajadores tuvieron derecho a asistencia médica, la posibilidad de ingresar a un hospital y recibir una pensión monetaria cuando no podían realizar sus labores ya sea a causa de un accidente o enfermedad. Extendiéndose dicha idea a otros países llegando a América del Sur en las primeras décadas del Siglo XX. Siendo los países pioneros Chile, Uruguay, Argentina y Brasil quienes instituyeron un régimen social en la década de 1920-1930. Posteriormente para los años 1930-1940 se introdujo en países como Bolivia, Colombia, Costa Rica, Ecuador, Paraguay, Perú y Venezuela. Siendo los países tardíos los de la región del Caribe y Centroamérica.\\

Entre los beneficios que la seguridad social brinda se encuentran los \textbf{planes de pensiones}, los cuales son convenios institucionales que protegen en la vejez, en la invalidez y en el caso de fallecimiento, del proveedor de sustento del hogar, a los dependientes quienes sufren la pérdida. Con el tiempo distintas instituciones privadas fueron implementando un programa similar para sus trabajadores o para un grupo específico, como lo proveen ciertos colegios de profesionales. 

\subsection{Antecendetes de la Seguridad Social en Guatemala}

La Seguridad Social en Guatemala no se inicia como tal sino hasta 1946 que el Congreso de la República de Guatemala sancionó el Decreto 295 "Ley Orgánica del Instituto Guatemalteco de Seguridad Social. Sin embargo pueden encontrarse algunos antecedentes de tipo laboral-social que velaron por el trabajador. Como primer punto de partida se tiene lo que son las "Leyes de Indias" que se remontan al tiempo de la colonia, compiladas en 1680, las cuales como un intento de la corona española establecían la limitación de abusos desmedidos que sus súbditos cometían contra el pueblo indígena. Estas fueron reformadas en las legislaturas estatales y republicanas en los años 1835 y 1851. Además de ello en 1877 se dictaron varias normas de tipo laboral.\\

En 1894 en que se dictó la "Ley de Trabajadores", sin embargo no se llegó a cumplir. Luego en 1906 nace la \textgravedbl Ley Protectora de Obreros sobre Accidentes de Trabajo\textacutedbl cuyas normas establecían prestaciones sociales a los trabajadores en caso de accidentes profesionales, asistencia médica en enfermedad y maternidad, además de subsidio monetario en caso de incapacidades y vitalicias en caso de incapacidad permanente. Sin embargo, dicha ley tuvo poca aplicación práctica. \\

En 1926 se decreta la \textgravedbl Ley de Trabajo\textacutedbl y en 1932 mediante un decreto se estableció el sistema de jubilaciones, pensiones y montepíos para funcionarios y empleados públicos cuya vigencia duró hasta 1970. Luego del movimiento revolucionario de octubre de 1944, con la promulgación de la Constitución Política de la República de Guatemala "se establece el seguro social obligatorio", creando así para 1946 "una institución autónoma, de derecho público, con personería jurídica propia y plena capacidad para adquirir derechos y contraer obligaciones, cuya finalidad es aplicar en beneficio del pueblo de Guatemala, un Régimen Nacional, Unitario y Obligatorio de Seguridad Social, de conformidad con el sistema de protección mínima".

\subsection{Antecedentes del Plan de Prestaciones de los Trabajadores al Servicio del Instituto Guatemalteco de Seguridad Social (IGSS)}

Para el año de 1969 se aprobó el Reglamento sobre Protección Relativa a Invalidez, Vejez y Sobrevivencia (Acuerdo 481 de la Junta Directiva) el cual sienta la base de lo que hoy se conoce como el Programa de Protección Relativa a Invalidez, Vejez y Sobrevivencia (Acuerdo 1124 de la Junta Directiva). Para el año 1970 el Acuerdo 498 establece la primera aplicación del Acuerdo 481 como plan piloto a todos los trabajadores del Instituto  con el objetivo de una ampliación regional, por lo que el Acuerdo 499 reglamentó la aplicación de lo acordado.\\

Posteriormente a petición de los trabajadores del Instituto con el objetivo de reducir los requisitos en los acuerdos 481, 498 y 499 y poder obtener un aumento en el monto de las pensiones, como una prestación patronal se emitió el Acuerdo 500 de la Junta Directiva el 23 de octubre de 1970. Luego en el año 1971 con el Acuerdo 512 el cual mediante estudios actuariales concluye el otorgar beneficios adicionales en un programa de protección social de carácter patronal. Para el año de 1974 se aprueba el Acuerdo 541 derogando los acuerdos 500 y 512, realizando algunas modificaciones a las prestaciones que en inicio se establecieron.\\


Es para el 3 de diciembre de 1990, derogando el Acuerdo 541, que se aprueba el Reglamento del Plan de Pensiones de los Trabajadores al Servicio del Instituto Guatemalteco de Seguridad Social (Acuerdo 905) en el cual se establece formalmente dicho programa, el cual es complementario al programa IVS (Acuerdo 1124), siendo de carácter contributivo por parte del patrono y de los trabajadores, cubriendo las prestaciones en caso de invalidez, vejez y sobrevivencia y cuota mortuoria, agregando la asignación única, además de permitir la contribución voluntaria en caso de termino de relación. Sufriendo las primeras modificaciones en el año 2002 por medio del acuerdo 1085 estableciendo una pensión máxima y modificando los requisitos para optar a una pensión de vejez.\\


Sin embargo en sentencia del 19 de noviembre de 2003, la Corte de Constitucionalidad declaró la inconstitucionalidad de algunos de los artículos del Acuerdo 1085, ordenando a la Junta Directiva del Instituto realizar las modificaciones necesarias para revertirla. Por lo que nace el Acuerdo 1135, modificando el mínimo de contribuciones y eliminando el monto de la pensión máxima. Posteriormente para el año 2016 se implementan modificaciones en el reglamento del Plan con la finalidad de salvaguardar la sostenibilidad financiera del mismo y garantizar el fin para el cual fue creado dicho plan, se estableció una pensión máxima en todas las pensiones, se modificaron los porcentajes de contribución, entre otras modificaciones.



%%%%%%%%%%%%%%%%%%%%%%%%%%%%%%%%%%%%%%%%%%%%%%%%%%%%%%%%%%%%%%%%%%%%%%%%%%%%%%%%%%%%%%%%%%%%%%%%%%%%%%%%%%%%%%%%%%%%%%%%%%%%%%%%%%%%%%%%%%%%%%%%
\newpage
\section{Generalidades}
Los planes de pensiones se denominan \textbf{sociales} cuando estos son administrados por el seguro social y \textbf{ocupacionales} cuando su administración corresponde a una entidad privada.  En Guatemala, los planes de pensiones ocupacionales son otorgados por algunas instituciones a sus trabajadores, como complemento de los beneficios brindados por el Insituto Guatemalteco de Seguridad Social (IGSS) a través del programa IVS. \\

Los planes de pensiones pueden clasificarse mediante el régimen al que estén asociados. Estos pueden ser de \textit{prestación definida} o \textit{cotización definida}. Un régimen se le llama de \textbf{prestación definida (PD)} cuando el monto de prestación se define mediante una fórmula, la cual es independiente del monto de cotización que el miembro aporta a lo largo de su carrera laboral. Los planes de \textbf{contribución definida (CD)} se caracterizan por que el monto de la pensión es calculado en el momento que el miembro del Plan califica a la pensión, con base a lo que contribuyó en su vida laboral. \\

Las contribuciones realizadas a un plan pueden ser capitalizadas en forma individual, para  cada miembro del Plan, o en forma colectiva para todos los miembros. Así un caso particular de este tipo de planes son los planes de \textbf{capitalización individual}. En un plan de pensiones las fuentes de financiamiento son:  contribuciones aportadas por los individuos participantes del plan, como también pueden ser en conjunto por los trabajadores y el empleador, o en el caso del sistema público pueden incluir subsidios gubernamentales ó impuestos destinados al seguro social.\\


Entre los financiamientos para los planes de pensiones ocupacionales, se encuentran los \textbf{métodos de costo individual}, los cuales determinan el equilibrio financiero del plan de los nuevos miembros como primer punto, considerando posteriormente, los ajustes necesarios para lograr el equilibrio del fondo destinado a la población inicial, por simplicidad se asume que todos los nuevos miembros ingresan al plan a una misma edad $b$ y se retiran a una edad $r$. Los \textbf{métodos de costos de beneficios acumulados}, se caracterizan por el suministro de una porción de la pensión final a beneficio de sus miembros en un intervalo de tiempo, también se pueden mencionar los \textbf{métodos de costo según edad de primera cotización}, los cuales buscan establecer el monto de contribución en función de la edad de ingreso al plan.\\
 

El sistema de financiamiento del Plan de Pensiones de los Trabajadores del Instituto Guatemalteco de Seguridad Social (IGSS) está conformado por el sistema de prima escalonada. Dicho plan posee una normativa de revisión actuarial anual con un período de equilibrio no menor a los 5 años. La finalidad del mismo consiste en otorgar como patrón, beneficios complementarios al programa de Invalidez, Vejez y Sobrevivencia (IVS) del Instituto como velador de la Seguridad Social. \\

El Plan es de carácter obligatorio para todos los trabajadores, contributivo por ambas partes (trabajador y el Instituto), complementario (como anteriormente se expuso) e independiente de los fondos y el manejo de cualquier otro programa que sea parte del Instituto. Este tipo de Plan, se encuentra entre los catalogados \textit{plan de pensiones ocupacional}. Posee un régimen de contribución definida y método de capitalización individual. Las prestaciones que el plan otorga para sus adeptos son: \textit{pensión en caso de invalidez, pensión en vejez, pensión en caso de sobrevivencia, asignación única} y \textit{cuota mortuoria}.

%%%%%%%%%%%%%%%%%%%%%%%%%%%%%%%%%%%%%%%%%%%%%%%%%%%%%%%%%%%%%%%%%%%%%%%%%%%%%%%%%%%%%%%%%%%%%%%%%%%%%%%%%%%%%%%%%%%%%%%%%%%%%%%%%%%%%%%%%%%%%%%%%%5
\newpage
\section{Las Prestaciones del Plan}

A continuación se listarán a detalle cada una de las prestaciones que el Instituto otorga a sus trabajadores mediante el Plan, especificando sus requisitos y los beneficios de las mismas. En lo que resta del documento se hará referencia, a no ser que se indique lo contrario, a los artículos del Reglamento del Plan de Pensiones de los Trabajadores al Servicio del Instituto Guatemalteco de Seguridad Social (IGSS) el cual es el Acuerdo 1135 de la Junta Directiva, cuya última modificación es el Acuerdo 1362 de la Junta Directiva.

\subsection{Invalidez}

Según el IGSS un asegurado, se considera en estado de \textit{invalidez} cuando presente una incapacidad para procurarse ingresos económicos como asalariado, en las condiciones en que los obtenía antes de la ocurrencia del riesgo que la originó. A continuación se presenta un esquema en el cual se desglosan los requisitos y los beneficios de la prestación que el Plan otorga en caso de invalidez a sus afiliados:

\begin{center}
	\begin{tabular}{|c||p{5cm}||p{7cm}|}
	\hline 
	\multicolumn{3}{|c|}{Prestación por Invalidez} \\ 
	\hline 
	\hline 
	No. & Requisitos & Beneficios \\ 
	\cline {1-1}
	\cline {2-2}
	\cline {3-3}
	1. & Ser declarado inválido por el IGSS. &   En el caso de invalidez parcial, se asigna una pensión igual a la mitad de la pensión de invalidez total, cuyo monto será un mínimo del 60\% correspondiente al último salario mensual o al porcentaje que corresponde según tabla del artículo 10.\\ 
	\hline 
	2. &  Haber contribuido 24 meses al Plan como mínimo en los últimos cuatro años inmediatamente anteriores al primer día de invalidez. & La pensión de gran invalidez, en caso de accidente, es igual al 80\% del último salario mensual del trabajador o al porcentaje que le corresponda según la tabla del artículo 10 del reglamento. \\ 
	\hline 
	3. & & La pensión de invalidez parcial se transforma en pensión de invalidez total al cumplir 55 años, mayor de 55 años se le dará pensión de invalidez total.\\
	\hline
\end{tabular} 
\end{center}

En ningún caso la pensión por invalidez puede ser mayor a los Q15,000.00.

\subsection{Vejez}

En un sistema de pensiones se encuentra el beneficio de una pensión vitalicia para las personas que se encuentran en una etapa de su vida correspondiente a la vejez, la cual se define según la RAE como "\textit{último periodo de la vida de una persona, que sigue a la madurez, en el cual se tiene edad avanzada}". A continuación se presenta un esquema en el cual se desglosan los requisitos y los beneficios de la prestación que el Plan otorga en la vejez a sus afiliados.


\begin{center}
	\begin{tabular}{|c||p{5cm}||p{6cm}|}
		\hline 
		\multicolumn{3}{|c|}{Prestación en la Vejez} \\ 
		\hline 
		\hline
		\rule[-1ex]{0pt}{2.5ex} No. & Requisitos &  Beneficios \\ 
		\hline 
		\rule[-1ex]{0pt}{2.5ex} 1. & Haber cumplido 55 años de edad. & Una pensión vitalicia cuyo monto se determina según los porcentajes que se presentan en la tabla del artículo 10. \\ 
		\hline 
		\rule[-1ex]{0pt}{2.5ex} 2. & Haber contribuido 240 meses al Plan. & El porcentaje de la pensión se aplica al último salario mensual. \\ 
		\hline 
		\rule[-1ex]{0pt}{2.5ex} 3. & Terminar su relación laboral con el Instituto. &  \\ 
		\hline 
	\end{tabular} 
\end{center} 

En todo caso la pensión de vejez no puede exceder los Q15,000.00.


\subsection{Sobrevivencia}

Según el Reglamento Sobre Protección Relativa a Invalidez, Vejez y Sobrevivencia (IVS)\footnote{Acuerdo 1124 de la Junta Directiva del Instituto Guatemalteco de Seguridad Social (IGSS).}, el estado de sobrevivencia es aquel en que quedan los beneficiarios dependientes económicos al fallecimiento del asegurado o pensionado.


El plan otorga pensiones a los beneficiarios \footnote{Persona a quien se extiende el derecho en el goce de los beneficios del Régimen de Seguridad Social, por razones de parentesco o de dependencia económica con el asegurado.} por muerte de miembros del plan. Los requisitos para que se le sea otorgada la pensión a un sobreviviente y sus beneficios se presentan a continuación:

\begin{center}
	\begin{tabular}{|c||p{6cm}||p{6cm}|}
		\hline 
		\multicolumn{3}{|c|}{Prestación en caso de Sobrevivencia} \\ 
		\hline
		\hline 
		No. & Requisitos & Beneficios \\ 
		\hline
		1. & A la fecha de fallecimiento el afiliado tuviese acreditados al menos 24 meses de contribución al Plan en los cuatro años inmediatamente anteriores. & El monto de la pensión mensual es igual a la que percibía el afiliado por vejez. \\ 
		\hline
		2. & A la fecha del fallecimiento el afiliado tuviese derecho o estuviere recibiendo pensión de invalidez o de vejez. & El monto de la pensión mensual es igual a la que correspondería por invalidez total. \\ 
		\hline 
		3. & Si la muerte ocurriera de accidente, se considera satisfecho el requisito de contribución del numeral (1). & La pensión se distribuye entre los beneficiarios con derecho, en las mismas proporciones previstas por el reglamento del programa IVS. \\ 
		\hline
	\end{tabular} 
\end{center}

En ningún caso la pensión por sobrevivencia puede ser mayor a los Q7,500.00.

\subsection{Asignación Única}

La asignación única es opcional para aquel trabajador que tenga los 55 años, pero sin tener derecho a pensionamiento por vejez. El único requisito que debe contemplarse es que dicho trabajador tenga acreditados por lo menos 12 meses de contribución efectiva. El monto de dicho beneficio será igual al 70\% del valor de las cuotas que hubiese aportado sin los intereses que estas pudieran haber generado.

\subsection{Cuota Mortuoria}

En caso de fallecimientos de miembros del plan se le asigna una cuota mortuoria por un total de Q1,500 para gastos de entierro. Dicha cuota se dará bajo las siguientes condiciones:

\begin{enumerate}
	\item Que el miembro tenga acreditados por lo menos dos meses de contribución en los últimos seis meses calendario.
	\item Que tenga derecho a pensión de invalidez o vejez.
	\item Sea pensionado por invalidez, vejez o sobrevivencia.
\end{enumerate}


\newpage
%%%%%%%%%%%%%%%%%%%%%%%%%%%%%%%%%%%%%%%%%%%%%%%%%%%%%%%%%%%%%%%%%%%%%%%%%%%%%%%%%%%%%%%%%%%%%%%%%%%%%%%%%%%%%%%%%%%%%%%%%%%%%%%%%%%%%%%%%%%%%%
\section{Las Contribuciones}

Las contribuciones ordinarias de la Institución (patronal) y los miembros del plan (laboral), se determinan en base a los porcentajes establecidos en el reglamento aplicados al salario mensual de los trabajadores.

\subsection{Salario Mensual del Trabajador}

Para efectos del cálculo de la Reserva Matemática en la siguiente sección, el salario mensual del trabajador se tomará como la doceava parte del salario ordinal y/o dieta anual, exceptuando bonificaciones, durante el año al que corresponda la valuación actuarial o el cálculo de la pensión, excepto para la pensión de vejez; que se entenderá como la treintaiseisava parte del salario ordinal y/o dieta devengado incluyendo cualquier otra retribución, exceptuando el Aguinaldo y las Asignación Complementaria Anual, durante los tres años anteriores a la fecha del acontecimiento que origina la pensión.

\subsection{Las Cuotas}

Las cuotas de contribución al plan se calculan mediante la doceava parte del salario ordinario y extraordinario que realice la institución, exceptuando  el Aguinaldo y las Asignación Complementaria Anual, en las siguientes proporciones:

\begin{center}
	\begin{tabular}{|c||c|}
		\hline 
		\rule[-1ex]{0pt}{2.5ex} Concepto & Porcentaje de Contribución ($porcon$)\\ 
		\hline 
		\hline 
		\rule[-1ex]{0pt}{2.5ex}
		\rule[-1ex]{0pt}{2.5ex} El instituto como patrono & 9.0 \\ 
		\hline 
		\rule[-1ex]{0pt}{2.5ex} Los miembros del plan & 4.5 \\ 
		\hline 
		\rule[-1ex]{0pt}{2.5ex} Total & 13.5 \\ 
		\hline 
	\end{tabular} 
\end{center}

En caso de que el trabajador termine su relación laboral con el instituto, éste tiene opción de continuar voluntariamente en el Plan, para lo cual debe cumplir con lo siguiente:

% \usepackage{array} is required
\begin{center}
	\begin{tabular}{|c||c||p{10cm}|}
		\hline 
		\multicolumn{3}{|c|}{Continuación Voluntaria en el Plan} \\ 
		\hline 
		\hline 
		\multirow{2}{24mm}[-20mm]{Requisitos}& 1. & Haber contribuido al Plan durante los últimos 10 años calendario. \\ 
		\cline{2-3}
		& 2. & Solicitar por escrito al Instituto, dentro de los tres meses siguientes a la finalización de su relación. \\ 
		\cline{2-3}
		& 3. & Pagar mensualmente las contribuciones, patronal y de miembro del Plan, en base a su último mes de contribución obligatoria. \\ 
		\hline 
	\end{tabular} 
\end{center}


%%%%%%%%%%%%%%%%%%%%%%%%%%%%%%%%%%%%%%%%%%%%%%%%%%%%%%%%%%%%%%%%%%%%%%%%%%%%%%%%%%%%%%%%%%%%%%%%%%%%%%%%%%%%%%%%%%%%%%%%%%%%%%%%%%%%%%%%%%%%%%%%
\newpage

\section{La Reserva Matemática (RM)}

El método de valuación actuarial que analiza la Reserva Matemática (RM) de un plan de pensiones se divide en dos, primero se calcula el valor presente esperado de los sueldos asegurados y segundo se calcula el valor presente esperado de las pensiones, de las prestaciones del plan, por pagar a los afiliados y a sus beneficiarios.\\

El cálculo del valor presente esperado ($V\!PE(x)$) de un capital $C$ es $V\!PE(C)=C\upsilon$, donde $\upsilon$ es el factor de descuento asociado a una tasa de interés anual efectiva $i$. 


\subsection{Cálculo del Valor Presente de la RM del Plan}

En el caso del Plan para facilitar los cálculos y dado que la tabla de datos que proporciona RRHH contiene solamente el salario del último mes (a la fecha de la consulta de la base de datos), se considerará este valor para el cálculo del monto de las pensiones.\\

Es necesario tener presente que regularmente los trabajadores tienen aumentos y no disminuciones de salario, por lo que utilizar en los cálculos el salario del último mes puede sobreestimar las obligaciones del Plan, por lo que esta simplificación corresponde a una posición conservadora en el cálculo de la RM.\\

Por otro lado el Plan paga las pensiones al final de cada mes, sin embargo los cálculos posteriores consideran un solo pago anual igual a 13 veces la pensión mensual en el caso de las pensiones por invalidez, vejez y sobrevivencia, pagaderas al inicio del año. Por lo que esto puede sobrestimar la obligación de las prestaciones  al no considerar los intereses que puede devengar este monto al hacerse los pagos en forma mensual.\\

Además se asume que las pensiones de cada año son pagadas en el caso de la sobrevivencia de los trabajadores pensionados por inválidez y por vejez, a solamente el riesgo de muerte y el VPE de las prestaciones por pagar a sobrevivientes, es calculado considerando la sobrevivencia del grupo de beneficiarios de cada afiliado, esto es hasta que el último de los beneficiarios se decremente.\\

Para el cálculo de los valores presentes esperados de las prestaciones que el plan ofrece se hará uso de la siguiente notación:\\
\begin{center} 
	\begin{tabular}{|c||l||c|}
		\hline 
		\rule[-1ex]{0pt}{2.5ex} No. & Concepto & Símbolo \\  
		\hline 
		\hline
		\rule[-1ex]{0pt}{2.5ex} 1. &Género. & $g$ \\ 
		\hline 
		\rule[-1ex]{0pt}{2.5ex} 2. & Edad. & $x$ \\ 
		\hline 
		\rule[-1ex]{0pt}{2.5ex} 3. & Tiempo de Servicio. & $s$ \\ 
		\hline 
		\rule[-1ex]{0pt}{2.5ex} 4. & Tasa de interés anual efectiva. & $i$ \\ 
		\hline 
		\rule[-1ex]{0pt}{2.5ex} 5. & Tasa promedio de incremento salarial anual esperado. & $j$ \\ 
		\hline 
		\rule[-1ex]{0pt}{2.5ex} 6. & El último sueldo mensual del trabajador. & $suemen$ \\ 
		\hline 
		\rule[-1ex]{0pt}{2.5ex} 7. & La pensión mensual del pensionado. & $penmen$ \\ 
		\hline 
	\end{tabular} 
\end{center}.

Donde $suemen$ corresponde al sueldo reportado en la información proporcionada por RRHH. A continuación se presentan las principales funciones a utilizar en el cálculo de los VPE de las distintas prestaciones que el Plan ofrece: \\

\begin{center}
	\begin{tabular}{|c||l||c|} %2
		\hline 
		\rule[-1ex]{0pt}{2.5ex} No. & Concepto & Representación \\ 
		\hline 
		\hline 
		\rule[-1ex]{0pt}{2.5ex} 1. & Factor de descuento asociado a la tasa de interés $i$. & $\upsilon$ \\ 
		\hline 
		\rule[-1ex]{0pt}{2.5ex} 2. & La proyección del sueldo anual de los trabajadores. & $salpro(suemen,k)$ \\ 
		\hline 
	\end{tabular} 
\end{center}.

Donde las funciones  $\upsilon$ y $salpro(suemen,k)$ están dadas por:\\

\begin{center}
	\begin{tabular}{ccc}
		$\upsilon = \dfrac{1}{1+i}$, & y & $salpro(suemen,k)=suemen(1+j)^{k}$
	\end{tabular} 
\end{center} 

En el caso de la función $salpro(suemen,k)$ se consideran únicamente los aumentos asociados a compensaciones por la inflación monetaria, de esta manera la tasa $j$ podría adoptarse en base a la inflación promedio esperada para los próximos años. Por lo tanto esta función no debe ser considerada como una componente de incrementos salariales relacionadas al merito.

Por otro lado, dado que el VPE de los prestaciones por pagar depende de la ocurrencia de algunos eventos, el momento en que estos ocurren y el monto de las prestaciones, es necesario la utilización de distribuciones de probabilidad, de las cuales se obtiene lo siguiente:\\


\begin{center}
	\begin{minipage}[t]{360pt} %3
	\begin{tabular}{|c||p{9cm}||c|}
	\hline 
	\rule[-1ex]{0pt}{2.5ex} No. & Concepto & Símbolo \\ 
	\hline 
	\hline 
	\rule[-1ex]{0pt}{2.5ex} 1. & La probabilidad de sobrevivencia a todos ($\tau$) los riesgos $k$ años en el futuro.  & $_{k}p^{(\tau)}$\footnote{Aplica solamente a los afiliados activos y que se considera dependiente de la edad, género y el tiempo de servicio.} \\ 
	\hline 
	\rule[-1ex]{0pt}{2.5ex} 2. & La probabilidad de sobrevivir $k$ años en el futuro. & $_{k}p^{(m)}$ \\ 		\hline 
	\rule[-1ex]{0pt}{2.5ex} 3. & La probabilidad de ocurrencia de muerte (m) durante el año $k$. & $q_{k}^{(m)}$ \\ 
	\hline 
	\rule[-1ex]{0pt}{2.5ex} 4. & La probabilidad de ocurrencia de invalidez (i) durante el año $k$. & $q_{k}^{(i)}$ \\ 
	\hline 
	\rule[-1ex]{0pt}{2.5ex} 5. & La probabilidad de ocurrencia de jubilación (j) durante el año $k$. & $q_{k}^{(j)}$ \\ 
	\hline 
	\rule[-1ex]{0pt}{2.5ex} 6. & La probabilidad de ocurrencia de retiro (r) durante el año $k$. & $q_{k}^{(r)}$ \\ 
	\hline 
\end{tabular} 
\end{minipage}
\end{center}


\subsubsection{Valor Presente Esperado de Las Contribuciones ($V\!PE(C)$)}

Para una simplificación de cálculos, se hará consideración que se realiza una sola contribución anual, pagadera al final del año en estudio, la cual es igual a 12 veces el sueldo mensual proyectado. Por lo tanto no se tomará en consideración los intereses que dichas contribuciones pueden generar al ser pagadas mes a  mes. Además se asume que cada año las contribuciones son pagadas solamente si los trabajadores sobreviven a todos los riesgos.\\

Primero se considera el porcentaje de contribución total $porcon$ el cual es igual al $13.5$ por ciento. De donde prosigue el cálculo del $V\!PE(C)$ de la siguiente forma:

\begin{equation*}
V\!PE(C)=\sum \limits_{k=0}^\omega \upsilon^{-k}\;_{k}p^{(\tau)}\;conanu(k)
\end{equation*}

Donde $\omega$ es el límite superior en años del intervalo de tiempo en estudio. Por otro lado la función $conanu(k)$ corresponde a la contribución anual al plan la cual está dada por:


\begin{equation*}
conanu(k)=12\;porcon\;sal\!pro(suemen,k)
\end{equation*}

\subsubsection{Valor Presente Esperado de La Pensión por Invalidez ($V\!PE(I)$)}

En el cálculo del $V\!PE(I)$ según los reglamentos el porcentaje atribuido a la pensión de invalidez está determinado según sea el tipo (gran, total y parcial). Sin embargo para una simplificación, dado el uso de tasa de invalidez general, se asumirá el mayor porcentaje posible, el cuál está restringido a un 80\% en forma conservadora. Seguido se encuentran las funciones particulares de la pensión de invalidez:\\

\begin{center}
	\begin{tabular}{|c||p{8cm}||c|}%4 
		\hline 
		\rule[-1ex]{0pt}{2.5ex} No. & Concepto & Función \\ 
		\hline 
		\hline 
		\rule[-1ex]{0pt}{2.5ex} 1. & Porcentaje dado por la tabla del artículo 10. & $tasa(x,s)$ \\ 
		\hline 
		\rule[-1ex]{0pt}{2.5ex} 2. & El porcentaje correspondiente a la pensión por invalidez. &  $porinv(x,s)$ \\ 
		\hline 
		\rule[-1ex]{0pt}{2.5ex} 3. & El VPE de una renta vitalicia anticipada inmediata anual, pagadera a una persona de edad $x$. &$\ddot{a}^{(i)}_{x+k}$\\
		\hline
		\rule[-1ex]{0pt}{2.5ex} 4. & La pensión anual en el año $k$ esperada al invalidarse. & $penanuinv(suemen,k)$ \\ 
		\hline 
		\rule[-1ex]{0pt}{2.5ex} 5. & El beneficio de la prestación en el año $k$. & $beninv(k)$ \\ 
		\hline 
	\end{tabular}
\end{center}

En donde las funciones están dadas por:\\
\[porinv(x,s)=max\{80, tabla(x,s)\},\]
\[penanuinv(suemen,k)=13\;porinv\;salpro(suemen,k),\]
\[ beninv(k)=penanuninv(suemen,k)\;\ddot{a}^{(i)}_{x+k}.\]


Para el VPE de $\ddot{a}^{(i)}_{x+k}$, se considera únicamente el riesgo de muerte en la población de los trabajadores inválidos. Prosiguiendo con el cálculo del $V\!PE(I)$ se hará uso de la siguiente:

\begin{equation*}
V\!PE(I)=V\!PE(I)_{A}+V\!PE(I)_{P},
\end{equation*}\\

donde $V\!PE(I)_{A}$ detona el $V\!PE(I)$ de los afiliados activos y $V\!PE(I)_{P}$ el $V\!PE(I)$ de los afiliados pasivos, las cuales están dadas por:

\begin{center}
	\begin{tabular}{ccc}
		$V\!PE(I)_{A}=\sum\limits_{k=0}^\omega\;\upsilon^{-k}\;_{k}p^{(t)}\;q^{(i)}_{k}\;beninv(k)$ & y & $V\!PE(I)_{P}=13\;penmen\;\ddot{a}^{(i)}_{x}$.
	\end{tabular} 
\end{center}

\subsubsection{Valor Presente Esperado de La Pensión por Vejez ($V\!PE(J)$)}

Al igual que en el cálculo del $V\!PE(I)$ se designarán funciones propias del valor presente esperado asociado a la pensión por vejez.

\begin{center}
	\begin{tabular}{|c||p{7cm}||c|} %5
		\hline 
		\rule[-1ex]{0pt}{2.5ex} No. & Concepto & Función \\ 
		\hline 
		\hline
		\rule[-1ex]{0pt}{2.5ex} 1. & Porcentaje de la tabla del artículo 10. & $porjub(x,s)$ \\ 
		\hline 
		\rule[-1ex]{0pt}{2.5ex} 2. & El beneficio de la prestación por vejez en el año $k$. &  $benjub(k)$ \\ 
		\hline 
		\rule[-1ex]{0pt}{2.5ex} 3. & El VPE de una renta vitalicia anticipada inmediata anual, pagadera a una persona de edad $x$. & $\ddot{a}^{(j)}_{x+k}$\\
		\hline
		\rule[-1ex]{0pt}{2.5ex} 4. & La pensión por vejez anual en el año $k$ esperada. & $penanujub(suemen,k)$ \\ 
		\hline 
	\end{tabular} 
\end{center}

En donde las funciones están dadas por:
\[penanujub(suemen,k)=13\;porjub\;salpro(suemen,k),\]
\[benjub(k)=penanujub(suemen,k)\;\ddot{a}^{(j)}_{x+k}.\]

Para el VPE de $\ddot{a}^{(j)}_{x+k}$, se considera unicamente el riesgo de muerte en la población de los trabajadores pensionados por vejez. Prosiguiendo con el cálculo del $V\!PE(J)$ se hará uso de la siguiente:

\begin{equation*}
V\!PE(J)=V\!PE(J)_{A}+V\!PE(J)_{P},
\end{equation*}

donde $V\!PE(J)_{A}$ detona los afiliados activos y $V\!PE(J)_{P}$ los afiliados pasivos, las cuales están dadas por:

\begin{center}
	\begin{tabular}{ccc}
		$V\!PE(J)_{A}=\sum\limits_{k=0}^\omega\;\upsilon^{-k}\;_{k}p^{(t)}\;q^{(j)}_{k}\;benjub(k)$ & y & $V\!PE(J)_{P}=13\;penmen\;\ddot{a}^{(j)}_{x}$.
	\end{tabular} 
\end{center}

\subsubsection{Valor Presente Esperado de La Pensión por Sobrevivencia ($V\!PE(S)$)}

En el caso de la pensión por sobrevivencia, se tienen tres grupos de quienes depende el $V\!PE(S)$, los cuales son los \textit{afiliados activos}, quienes al momento de fallecer sus beneficiarios reciben el equivalente a una pensión por invalidez según corresponda, los\textit{ afiliados pasivos}, quienes al momento de fallecer sus beneficiarios reciben el equivalente a la pensión que recibían y los \textit{fallecidos}, cuyos beneficiarios ya se encuentran recibiendo la pensión correspondiente.

Para el cálculo de $V\!PE(S)$ será necesario el uso de las siguientes:
\begin{center}
	\begin{tabular}{|c||p{8cm}||c|} %6
		\hline 
		\rule[-1ex]{0pt}{2.5ex}No. & Concepto & Función \\ 
		\hline 
		\hline 
		\rule[-1ex]{0pt}{2.5ex} 1. & La probabilidad de sobrevivencia del grupo de beneficiarios. & $_{k}p^{(b)}$ \\ 
		\hline 
		\rule[-1ex]{0pt}{2.5ex} 2. & La pensión anual esperada del grupo de sobreviventes de los afiliados activos. & $penanusob_{A}(suemen,k)$\\
		\hline 
		\rule[-1ex]{0pt}{2.5ex} 3. & La pensión anual del grupo de sobrevivientes de los afiliados pasivos. & $penanusob_{P}$ \\ 
		\hline 
		\rule[-1ex]{0pt}{2.5ex} 4. & La pensión anual del grupo de sobrevivientes de los afiliados fallecidos. & $penanusob_{F}$ \\ 
		\hline 
		\rule[-1ex]{0pt}{2.5ex} 5. & El beneficio de la prestación a sobrevivientes de afiliados activos. & $bensob_{A}(k)$ \\ 
		\hline 
		\rule[-1ex]{0pt}{2.5ex} 6. & El beneficio de la prestación a sobrevivientes de afiliados pasivos. & $bensob_{P}(k)$ \\ 
		\hline 
	\end{tabular} 
\end{center}

En donde las funciones están dadas por:
\[penanusob_{A}(suemen,k)=13\;porinv\;sal\!pro(suemen,k),\]
\[bensob_{A}(k)=\sum\limits_{k=0}^\omega\;\upsilon^{-k}\;_{k}p^{(b)}\;penanusob_{A}(suemen,k),\]
\[penanusob_{P}=13\;penmen,\]
\[bensob_{P}(k)=\sum\limits_{k=0}^\omega\;\upsilon^{-k}\;_{k}p^{(b)}\;penanusob_{P},\]
\[penanusob_{F}=13\;penmen,\]
 
y para el cálculo de la probabilidad de sobrevivenia del grupo de beneficiarios de cada afiliado, se consideran los siguientes riesgos:

\begin{itemize}
	\item [*] Muerte, para el caso del cónyuge del afiliado.
	\item [*] Sobrevivencia a la mayoría de edad, para los hijos del afiliado.
\end{itemize}

Prosiguiendo con el cálculo del $V\!PE(S)$ se hará uso de la siguiente ecuación:
\[V\!PE(S)=V\!PE(S)_{A}+V\!PE(S)_{P}+V\!PE(S)_{F},\]

en donde $V\!PE(S)_{A}$ representa el $V\!PE(S)$ de los afiliados activos, $V\!PE(S)_{P}$ el de los afiliados pasivos y $aV\!PE(S)_{F}$ el de los afiliados fallecidos, las cuales están dadas por:

\[V\!PE(S)_{A}=\sum\limits_{k=0}^\omega\!\upsilon^{-k}\!_{k}p^{(\tau)}\!q_{k}^{(m)}\!bensob_{A}(k),\]
\[V\!PE(S)_{P}=\sum\limits_{k=0}^\omega\!\upsilon^{-k}\!_{k}p^{(m)}\!q_{k}^{(m)}\!bensob_{P}(k),\]
\[V\!PE(S)_{F}=\sum\limits_{k=0}^\omega\!\upsilon^{-k}\!_{k}p^{(b)}penanunsob_{F}\]


\subsubsection{Valor Presente Esperado de La Asignación Única ($V\!PE(AU)$)}

El Plan paga este beneficio prácticamente al momento en que el trabajador se retira, sin embargo el cálculo siguiente considera que el pago se haga al inicio del año del retiro, incluyendo los 12 sueldos del mismo, por lo que puede sobreestimar la obligación de esta prestación al no considerar los intereses que puede devengar este monto al hacerse el pago más adelante, como también puede que el trabajador se retire antes de finalizar el año. \\

Para el cálculo del  $V\!PE(AU)$ será necesario el uso de una única función la cual representa el beneficio de la prestación la cual está dada por:
\[benasi(k)=(0.7)(0.3)\sum\limits_{l=0}^{k}\;12\;sal\!pro(suemen,l).\]
 
Calculándose entonces el $VPE$ como:
\[ V\!PE(AU)=\sum\limits_{k=0}^{\omega}\!\upsilon^{-k}\!_{k}p^{(\tau)}\!q_{k}^{(r)}\!benasi(k)\]

Cabe destacar que el pago del beneficio está condicionado por el retiro del afiliado del Plan, cuya probabilidad ($q_{k}^{(r)}$) es dependiente de las características del afiliado (edad, genero, tiempo de servicio) y para la tasa de retiro del Plan debe comprender los dos riesgos que pueden ser los causantes de la baja del trabajador: \textit{el despido} y \textit{la renuncia}.

\subsubsection{Valor Presente Esperado de La Cuota Mortuoria ($V\!PE(CM)$)}

Este valor puede ser calculado para cada grupo de miembros del Plan (activos y pensionados por vejez o invalidez), considerando las diferentes tasas de mortalidad de cada grupo. El Plan paga este beneficio prácticamente al momento del fallecimiento del trabajador, sin embargo en el cálculo siguiente se considera que el pago se hace al final del año del fallecimiento por lo que se puede subestimar la obligación de esta prestación al no considerar los intereses que puede devengar este monto, al hacerse los pagos en forma mensual.\\

Para el cálculo del $V\!PE$ de esta prestación se utiliza el $V\!PE$ del pago de Q1.00 al final del añor en que fallece el miembro del Plan, el cual depende únicamente de la edad del trabajador que se representa por $A_{x}$. Calculándose entonces el $V\!PE(CM)$ como:
\[ V\!PE(CM)=1,500\;A_{x} \]

\paragraph{Nota} Modelo realizado con base al estudio proporcionado por el Dr. Roberto Molina \cite{pla0105}, actualizado según las modificaciones realizadas al Plan de Pensiones de los Trabajadores al Servicio del Instituto Guatemalteco de Seguridad Social (IGSS). Para mayor información sobre las fórmulas actuariales de los $V\!PE$ de los pagos anticipados o diferidos de anualidades y probabilidades se puede consultar (\cite{amssp}, pp. 93-99) y (\cite{tass}, pp. 47-114).

\subsection{Cálculo de la Prima Media General (PMG)}

El sistema de financiamiento de la PMG se basa en el concepto de un monto de contribución (prima) constante durante el tiempo de gestión del plan de pensiones, el cual equilibra la diferencia entre el VPE de las prestaciones totales esperadas en el futuro y las reservas iniciales con los ingresos de las contribuciones. \\


Para el cálculo de la PMG, se calcula una prima ponderada que depende de la prima media de la población inicial (PM1) y de la Prima media de la población futura (PM2). Teóricamente, se estima que la PM1 sea mayor que la PM2. Para calcular las PM1, PM2 y PMG, se hará uso de la siguiente notación:

\begin{center}
	\begin{tabular}{|c||p{12cm}|}
		\hline 
		Símbolo & Concepto \\ 
		\hline 
		\hline 
		$Ac(x,0)$ & Es la población inicial de edad x para la fecha de la valuación. \\ 
		\hline 
		$s(x,0)$ & Es el promedio de los salarios asegurados de la población. \\ 
		\hline 
		$pr(x)$ & Es la proporción de nuevos participantes que ingresan a una edad x. \\ 
		\hline 
		$np(t)$ & Es la cantidad de nuevos participantes que ingresan en año t. \\ 
		\hline 
		$ns(x)$ & Es el promedio de los salarios asegurados de los nuevos parcipantes para la fecha de valuación. \\ 
		\hline 
		$V\!PES1$ & Es el valor presente esperado de lo salarios de la población inicial. \\ 
		\hline 
		$V\!PEP1$ & Es el valor presente esperado de todas las prestaciones de la población inicial. \\ 
		\hline 
		$V\!PES2$ & Es el valor presente esperado de los salarios de los nuevos participantes. \\ 
		\hline 
		$V\!PEP2$ & Es el valor presente esperado de todas las prestaciones de los nuevos partcipantes. \\ 
		\hline
	\end{tabular} 
\end{center}

Prosiguiendo de la siguiente manera:

\[ PM1=\dfrac{\sum\limits_{x=b}^{r-1}Ac(x,0)s(x,0)V\!PEP1(x)}{\sum\limits_{x=b}^{r-1}Ac(x,0)s(x,0)V\!PES1(x)} \]
\[ PM2=\dfrac{\sum\limits_{x=b}^{r-1}pr(xp)ns(x,0)V\!PEP2(x)}{\sum\limits_{x=b}^{r-1}pr(x)ns(x,0)V\!PES2(x)}. \]

Reescribiendo los gastos de prestaciones e ingresos de la población inicial como GP1 e IN1 respectivamente y de los nuevos participantes como GP2 e IN2, se tiene entonces lo siguiente:

\begin{center}
	$ PM1=\dfrac{GP1}{IN1} $      y       $ PM2=\dfrac{GP2}{IN2} $.
\end{center} 

Además defínase un nuevo factor $k$ dado por:
\begin{center}
	\[ k=\sum\limits_{t=1}^{\infty}na(t)\upsilon^{t-0.5} \],
\end{center}

de donde puede entonces calculares PMG como:
\begin{center}
	\[ PMG=\dfrac{GP1+k\;GP2}{IN1+k\;IN2} \].
\end{center}

Por otro lado si se cuenta con un valor inicial de la reserva del Plan, se toma como año inicial $t=0$ ($RES(0)$) aquel en el cual se realiza dicho cálculo, tomando en consideración la proyección de los gastos e ingresos de modo que el valor presente al tiempo $t=0$ del gasto total (prestaciones y administrativo) menos los ingresos (exceptuando los ingresos relacionados a las inversiones) ($V\!P(GT-I)$) y el valor presente al tiempo $t=0$ del total de ingresos sujeto a las contribuciones ($V\!P(TISC)$) durante un intervalo de tiempo $(0, \infty)$ se puede entonces calcular la PMG como:

	\[ PMG=\dfrac{V\!P(GT-I)-RES(0)}{V\!P(TISC)}. \]
	
\paragraph{Nota}Modelos extraídos de  (\cite{amssp}, pp. 88-89) y (\cite{fsp}).


%%%%%%%%%%%%%%%%%%%%%%%%%%%%%%%%%%%%%%%%%%%%%%%%%%%%%%%%%%%%%%%%%%%%%%%%%%%%%%%%%%%%%%%%%%%%%%%%%%%%%%%%%%%%%%%%%%%%%%%%%%%%%%%%%%%%%%%%%%%%%%%%%%

\newpage
\section{El Flujo de Efectivo}

En el presente capítulo se hará una discusión teórica de las bases y la metodología en la realización de una valuación actuarial utilizando el método del análisis del flujo de efectivo, el cual permite calcular la Longitud de Período de Equilibrio (LPE). El método del análisis del flujo, el cuál se refiere a la técnica de proyección, que es la más recomendable para un sistema de pensiones, cuyo sistema de financiamiento es la Prima Escalonada.\\

Para la técnica de proyección se hace uso del método de componente, el cual consiste en la clasificación de la población cubierta entre componentes y simular su evolución a lo largo del tiempo de vida del esquema de pensiones. La extensión de la subclasificación dependerá de la disponibilidad de datos proporcionados, en este caso, por el área de RRHH del Instituto. Sin embargo la subclasificación mínima requerida para la realización de una valuación es por el desglose de las categorías de las personas pensionadas (por vejez, invalidez y sobrevivencia), subdividiendo cada una por hombres y mujeres clasificados por edad individual.

\subsection{Proyecciones Demográficas}

Como punto de partida para la técnica de proyección es la elaboración de estimadores, que indiquen el número de población en cada uno de los subgrupos del método de componentes, que en este caso serían: miembros del plan activos y pensionados (por vejez, invalidez y sobrevivencia), en puntos específicos de tiempo $(t=1, 2,...)$ partiendo de un tiempo inicial $t=0$, siendo este el año correspondiente al estudio de la valuación actuarial.\\

El procedimiento de la proyección demográfica puede considerarse iterativa, mediante una operación multiplicativa de matrices: 

\[ n_{t}=n_{t-1}Q_{t-1},\]

donde $n$, es un vector fila, cuyos elementos representan los valores de proyecciones demográficas en el tiempo $t$ y $Q_{t-1}$ es una matriz de transición para el intervalo de tiempo $(t-1,t)$ dada por:

\[ Q_{t}= \left( \begin{array}{ccccc}\ p^{(aa)} & q^{(ar)} & q^{(ai)} & q^{(aw)} & q^{(ao)}\\
0 & p^{(rr)} & 0 & q^{(rw)} & q^{(ro)}\\
0 & 0 & p(ii) & q^{(iw)} & q^{(io)}\\
0 & 0 & 0 & p^{(ww)} & 0\\
0 & 0 & 0 & 0  & p^{(oo)}\end{array}\right) ,\]

en donde se utiliza la siguiente nomenclatura:

\begin{center}
	\begin{tabular}{|c||c|}
	\hline 
	\rule[-1ex]{0pt}{2.5ex} Símbolo & Concepto \\ 
	\hline 
	\hline 
	\rule[-1ex]{0pt}{2.5ex} $a$ & Miembros activos. \\ 
	\hline 
	\rule[-1ex]{0pt}{2.5ex} $r$ & Miembros que entran al estado de vejez. \\ 
	\hline 
	\rule[-1ex]{0pt}{2.5ex} $i$ & Miembros  que entran al estado de invalidez. \\ 
	\hline 
	\rule[-1ex]{0pt}{2.5ex} $w$ & Cónyuges de miembros que entran en estado de viudez. \\ 
	\hline 
	\rule[-1ex]{0pt}{2.5ex} $o$ & Huérfanos de miembros del plan. \\ 
	\hline 
	\rule[-1ex]{0pt}{2.5ex} $p^{(rr)}$ & La probabilidad de permanecer en el mismo estado r. \\ 
	\hline 
	\rule[-1ex]{0pt}{2.5ex} $q^{(rs)}$ & La probabilidad de transición del estado r al estado s. \\ 
	\hline 
\end{tabular}
\end{center} 

Para poder ser aplicable el procedimiento anterior, es necesaria la subclasificación de la población por edad y sexo, además sería preferible que la población activa esté subclasificada por servicio pasado. Entonces el procedimiento se aplica a la subclasificación más baja del esquema y los resultados se agregan dando varios subtotales y totales. Para el caso de sobrevivientes, se requiere un procedimiento adicional, después de cada iteración, el cual consiste en clasificar a los nuevos cónyuges en estado de viudez y huérfanos derivados de la muerte de miembros del Plan a la edad $x$ según la edad del cónyuge viudo $(y)$ o del huérfano $(z)$ antes de proceder con la siguiente iteración. \\

\subsubsection{Recolección de datos}

Los datos a la fecha de valuación, que dan el desglose de edad-sexo de cada una de las subpoblaciones (miembros activos, pensionados por vejez, invalidez y sobrevivientes: viudas/viudos inválidos y huérfanos), proporcionan el punto de partida para el procedimiento de iteración. Por simplificación los procedimientos de esta sección se refieren a la definición de edad más cercana. Dado que la valuación actuarial se realiza con información observada del año anterior se harán supuestos de la siguiente manera: los datos relativos a los pensionados se refieren a aquellos que acreditan el derecho a una pensión y que los datos relativos a los miembros activos se refieren a aquellos que han acreditado al menos una cotización en el año referente al estudio de la valuación.\\

Dado que la técnica de proyección asume que la población asegurada manifiesta una dinámica constante, las estimaciones se hacen con respecto a los futuros miembros del sistema, por lo que se realiza de la siguiente forma: las indicaciones de la tasa esperada de crecimiento, del total de la población activa asegurada, junto con la distribución relativa de sexo y edad correspondiente a los nuevos participantes son dadas. 

\subsubsection{Fórmulas para la proyección demográfica}

A partir de los datos de población en la fecha de la valuación $(t=0)$ y de las probabilidades de transición se aplican a las proyecciones por sexo y edad. En el caso de la proyección de la población activa, nuevos miembros del año inmediatamente anterior deben incorporarse antes de proceder a la siguiente iteración. A continuación se indican las fórmulas de proyección para la población de miembros activos, el método de proyección de la población de pensionados se ejemplifica con referencia a los pensionados por vejez, haciendo uso de la siguiente notación:\\

\begin{center}
	\begin{tabular}{|c||p{11cm}|}
	\hline 
	\rule[-1ex]{0pt}{2.5ex} Símbolo & Concepto \\ 
	\hline 
	\hline 
	\rule[-1ex]{0pt}{2.5ex} $Act(x,s,t)$ & Población activa de edad $x$, con curtósis de servicio de $s$ años al tiempo $t$. \\ 
	\hline 
	\rule[-1ex]{0pt}{2.5ex} $Ac(x,t)$ & Denota a la población activa de edad $x$ en el tiempo $t$. Las poblaciones correspondientes a las prestaciones  se denotan por $Re(x,t)$, $In(x,t)$ y $Wi(x,t)$. \\ 
	%\hline 
	%\rule[-1ex]{0pt}{2.5ex} $Re$ &  \\ 
	%\hline 
	%\rule[-1ex]{0pt}{2.5ex} $In$ &  \\ 
	%\hline 
	%\rule[-1ex]{0pt}{2.5ex} $Wi$ &  \\ 
	\hline 
	\rule[-1ex]{0pt}{2.5ex} $A(t)$ & Denota la población activa en el tiempo $t$. Las poblaciones correspondientes a las prestaciones se denotan por $Re(t)$, $In(t)$ y $Wi(t)$.  \\ 
	\hline 
	\rule[-1ex]{0pt}{2.5ex} $N(x,t)$ & El número de nuevos participantes de edad x en la proyección del año t. \\ 
	\hline 
\end{tabular} 
\end{center}
Teniendo en cuenta que $b\leqslant x<r, s>0$, siendo $b$ la edad de entrada al Plan más joven, $r$ la edad de entrada a vejez.

\paragraph{Proyección de Población Activa}

La proyección de la población activa del tiempo $t-1$ al tiempo $t$ se expresa mediante:

\[ Act(x,s,t)=Act(x-1,s-1,t-1)\;p^{(aa)}_{x-1} \]

donde $b+1\leq x<r$, $s\geq1$. Sea $dp$ la densidad de prestaciones, la proporción del período potencial de servicio en el intervalo de edad (x-1,x), entonces la ecuación anterior se transforma en:

\[Act(x,s,t)=[dp\;Act(x-1,s-1,t-1)+(1-dp)\;Act(x-1,s,t-1)]\;p_{x-1}^{(aa)}\]

La tasa de incremento de la población asegurada total activa $\rho(t)$ es dada, de donde la proyección de la población activa total para un año $t$ es:

\[ A(t)=A(t-1)(1+\rho(t)) \]

en donde la distribución de edad proporcionada de nuevos entrantes por edad al ingreso, $pr(x)$ es dada, entonces $Z(x,t)$, los sobrevivientes activos en un tiempo $t$ de los nuevos miembros durante el año $(t-1,t)$, se estima como:

\begin{equation*}
Z(x,t)=\dfrac{A(t)-\sum\limits_{y}\left( \sum\limits_{s\geq1}Act(y,s,t)+(1-dp)Act(y-1,0,t-1)p^{(aa)}_{y-1}\right) }{\sum\limits_{y}pr(y)p^{(aa)}_{y-0.5:0.5}}\;pr(x)p^{(aa)}_{x-0.5:0.5} 
\end{equation*}

Para la proyección de $Act(x,0,t)$ se tiene:

\[ Act(x,0,t)=Z(x,t)+(1-dp)Act(x-1,0,t-1)p^{(aa)}_{x-1} \]

Se supone que los nuevos miembros deben ingresar a mediados del año, de donde el número actual de nuevos participantes a la edad $x$ puede estimarse mediante la regresión como:

\[ N(x,t)=\frac{Z(x,t)}{p^{(aa)}_{x-0.5:0.5}} \]

\paragraph{Proyección de la población pensionada}
El procedimiento para la proyección de las subclasificaciones de pensionados se ilustrará mediante los que corresponden a los miembros del Plan que reciben pensión por vejez.

Para la población de pensionados por vejez de edad $x$ en un tiempo $t$:

\[ Re(x,t)=Re(x-1,t-1)p^{(rr)}_{x-1}+Ac(x-1,t-1)q_{x-1}^{(ar)}+N(x,t)q^{(ar)}_{x-0.5:0.5} \]

donde el factor $q_{x-0.5:0.5}^{(ar)}$ es análogo a $q_{x}^{(ar)}$ pero relacionado a un intervalo de edad $(x-0.5,0.5)$.

Para la población total de pensionados por vejez:

\[ R(t)=\sum\limits_{x}Re(x,t) \]

\paragraph{Nota}Modelo extraído de (\cite{amssp}, pp. 65-73). 

\subsection{Proyecciones Financieras}
Como siguiente paso, una vez realizadas las proyecciones demográficas necesarias, se procede a estimar el gasto total del salario anual asegurado y de los importes totales anuales de las subclasificaciones de pensiones en curso en un punto específico de tiempo $(t=1,2,...)$ partiendo de un tiempo inicial $t=0$. Los importes medios se calculan año tras año en paralelo con la proyección demográfica correspondiente. Un promedio per cápita se calcula para cada subclasificación de la población generada por la proyección demográfica. 

El método para la modelación de la proyección financiera, comienza con una variación en la estructura salarial promedio relacionada con la edad en un intervalo de tiempo y luego calcula los salarios promedios según edad y tiempo que permiten la escalada general de los salarios. Además modela la distribución de los salarios por edad.


\subsubsection{Recolección de Datos}

Se supone que el salario potencial corresponde al indicado como salario asegurado en el reglamento del Plan. Para la aplicación de la proyección financiera se requiere una mínima distribución salarial a cada edad de la población. Alternativamente se proporciona el coeficiente de variación de esta distribución denotado por $cv(x)$. En el caso de los pensionados existentes a la fecha de la valuación, se requerirán importes anuales medios de pensión en vigor por sexo a cada edad y para cada subclasificación. Otro dato para el análisis financiero, es el fondo de reserva acumulado en la fecha de valuación denotado por $Fd(0)$.


\subsubsection{Base actuarial}

Para las bases de las proyecciones financieras se especifican funciones que dependen de edad y tiempo, que son específicos del sexo.

\begin{tabular}{|c||p{11cm}|}
	\hline 
	Símbolo & Concepto \\ 
	\hline
 	\hline
	$j(t)$ & Cambio en la estructura salarial promedio por edad. \\ 
	\hline 
	$\gamma(t)$ & La tasa de escalada salarial en cada año de proyección \\ 
	\hline 
	$\beta(t)$ & La tasa de indexación de las pensiones a cada año de proyección. \\ 
	\hline 
	$i(t)$ & La tasa de rendimiento de la inversión a cada año de proyección. \\ 
	\hline 
	$dc(t)$ & La fracción de año durante el cual las contribuciones son efectivamente pagaderas (densidad de distribución). \\ 
	\hline 
	$dp(x)$ & La fracción del período de servicio potencial que se calculará efectivamente con fines de pensiones.\\
	\hline 
\end{tabular} 

\subsubsection{Fórmulas de proyección financiera}

\paragraph{Proyección de los salarios medios}

Sea $ss(x,0)$ la función salarial relativa inicial, que indica los niveles relativos de los salarios medios por edad en $t=0$, se establece expresando el salario promedio a la edad $x$ como índice, digamos con 1000 participantes en la edad más joven $b$, como sigue:

\[ ss(x,0)=1000\frac{s(x,0)}{s(b,0)} \]

La función de salario relativo para el año $t$ de proyección es expresada por:

\[ ss(x,t)=ss(x-1,t)\left( \dfrac{ss(x,0)}{ss(x-1,0)}\right) ^{j(t)} \]

Un factor de ajuste $j(t)$ para un valor mayor (menor) que la unidad implica un ensanchamiento (estrechamiento) de la variación del salario medio por edad, si $j(t)=0$ el salario medio se convierte en el mismo en todas las edades. Por otro lado el salario medio a una edad $x$ en el año de proyección $t$ está dado por:

\[ s(x,t)=ss(x,t)(1+\gamma(t))\dfrac{\sum\limits_{b}^{r-1}s(y,t-1)Ac(y,t-1)}{\sum\limits_{b}^{r-1}ss(y,t)Ac(y,t)}\dfrac{\sum\limits_{b}^{r-1}Ac(y,t)}{\sum\limits_{b}^{r-1}Ac(y,t-1)} \]

donde $Ac(y,t)$ denota la proyección de la población activa a una edad $y$ en el tiempo $t$. El gasto total del salario asegurado total en curso en un tiempo $t$ será estimado como:

\[ \sum\limits_{x}Ac(x,t)s(x,t)dc(x) \] 

\paragraph{Proyección de la distribución de salarios}

Se hace el supuesto que la distribución de los salarios en cualquier edad es $lognormal$. Sea $y$ el salario y sea $z=log_{e}(y)$. Sea $\mu_{y}$ y $\mu_{z}$ las respectivas medias y $\sigma^{2}_{y}$ y $\mu^{2}_{z}$ las respectivas varianzas. Entonces:

\[ \mu_{y}=e^{\mu_{z}+\frac{1}{2}\sigma^{2}_{z}} \]
\[ \sigma_{y}^{2}=e^{2\mu_{z}+\sigma^{2}_{z}}( e^{\sigma^{2}_{z}}-1) \]

El salario promedio ya se ha proyectado $\mu_{y}=s(x,t)$. Suponiendo que el coeficiente de variación de $y$ es invariante en el valor inicial $cv(x)$ entonces $\sigma_{y}=cv(x)s(x,t)$. Los parámetros de la distribución de $z$ se pueden calcular de la siguiente manera invirtiendo las ecuaciones anteriores:

\[ \mu_{z}=log_{e}\left( \dfrac{\mu_{y}}{\sqrt{1-cv(x)^{2}}}\right)  \]
\[ \sigma_{z}^{2}=log_{e}\left( 1+cv(x)^{2}\right)  \]

De igual manera, el salario promedio de la población entre dos niveles de salarios, digamos $y_{a}$ y $y_{b}$ está dado por:

\[ s(x,t)=\dfrac{\Phi(w_{b}-\sigma_{z})-\Phi(w_{a}-\sigma_{z})}{\Phi(w_{b})-\Phi(w_{a})} \]

donde $\Phi(w)$ es la función de distribución de variación normal estándar y $w_{a}=\frac{(log_{e}y_{a}-\mu_{z})}{\sigma_{z}}$ y $w_{b}=\frac{(log_{e}y_{b}-\mu_{z})}{\sigma_{z}}$. En la práctica, la población activa a cada edad, $Ac(x,t)$, se debe dividir en un número limitado de grupos según el nivel de sueldo, por ejemplo, se pueden utilizar tres grupos:

\begin{itemize}
	\item [1.] El grupo de bajos ingresos (el 30 por ciento más bajo de la población).
	\item [2.] El grupo de ingresos medios (el 40 por ciento medio).
	\item [3.] El grupo de ingresos altos (el 30 por ciento más alto).
\end{itemize}

Sean $s1(x,t)$, $s2(x,t)$ y $s3(x,t)$ la representación del promedio de salarios respectivos. Primero los valores $w1$ y $w2$ se determinan tal que $\Phi(w1)=0.3$ y $\Phi(w2)=0.7$. Los promedios de los salarios están dados por:

\[ s1(x,t)=s(x,t)\dfrac{\Phi(w1-\sigma_{z})}{\Phi(w1)} \]
\[ s2(x,t)=s(x,t)\dfrac{\Phi(w2-\sigma_{z})-\Phi(w1-\sigma_{z})}{\Phi(w2)-\Phi(w1)} \]
\[ s3(x,t)=s(x,t)=\dfrac{1-\Phi(w2-\sigma_{z})}{1-\Phi(w2)} \]

\paragraph{Proyección del gasto de prestaciones}

Sea $b(x,t)$ el monto promedio per cápita de la pensión de los pensionados por vejez de edad $x$ en el tiempo $t$ $(Re(x,t))$, se tiene que $Re(x,t)$ está constituido de la combinación de tres funciones distintas, por lo que el monto per cápita de cada una debe ser determinado y tomar un promedio ponderado. El monto de pensión per cápita del primer elemento se determina como:

\[ b(x,t)=b(x-1,t-1)(1+\beta(t)) \]

El segundo elemento se refiere a aquellos que se retiran del régimen a la edad $x-1$ un año anterior $t-1$, es decir $Ac(x-1,t-1)$. Siendo este calculado con base al servicio pasado promedio de las cohortes. Denotando $sv(x-1,t-1)$ el promedio de servicio de contribución de las personas $Ac(x-1,t-1)$ se tiene que el servicio después de un año de este grupo se proyectaría como:

\[ sv(x,t)=sv(x-1,t-1)+dp(x-0.5) \]

Bajo las suposiciones hechas, las pensiones por vejez toman lugar al final de cada año de edad. Para este método el segundo elemento de $Re(x,t)$ se constituye de tres subgrupos con salarios medios $s1(x,t)$, $s2(x,t)$ y $s3(x,t)$, por lo que se debe calcular el monto de la pensión de cada subgrupo y tomar un promedio ponderado. Si además se tiene una distribución de servicios pasados, $Ac(x,t)$ sería considerada como constituido por subgrupos de primer nivel $Act(x,s,t)$, de acuerdo con servicios pasados $s$, donde cada subgrupo de primer nivel está constituido por tres subrgrupos de segundo nivel según nivel de salario. Por lo que un monto de pensión percápita se calcularía para cada subgrupo de segundo nivel y luego se toma un promedio ponderado de todos los subgrupos de segundo nivel. \\

El tercer elemento de $Re(x,t)$ corresponde a las pensiones por vejez que surgen de los nuevos entrantes del año inmediatamente anterior $(t-1,t)$. Este grupo debería tener un servicio pasado de medio año o menos. El salario se tomaría como idéntico al del segundo elemento.

\paragraph{Nota}Modelo extraído de \cite{amssp} para un análisis más extenso puede consultarse \cite{apss}.


\subsection{Equilibrio Financiero}

Para determinar el indicador LPE del régimen, es necesario encontrar el equilibrio financiero del mismo, el cual se estudia mediante la evolución del fondo de reserva cada año. 

Las ecuaciones básicas para el cálculo del equilibrio financiero utilizan la siguiente notación:

\begin{center}
\begin{tabular}{|c||p{11cm}|}
	\hline 
	Símbolo & Concepto \\ 
	\hline 
	\hline 
	$V(t)$ & Reserva al final del año $t$. \\ 
	\hline 
	$R(t)$ & Incremento Anual total en el año $t$ (incluyendo las ganacias derivadas del interés). \\ 
	\hline 
	$C(t)$ & Ingreso anual de contribuciones en el año $t$ (excluyendo las ganacias generadas por el interés). \\ 
	\hline 
	$I(t)$ & Ingreso anual total de ganacias generadas por el interés. \\ 
	\hline 
	$B(t)$ & Costo anual en el año $t$. \\ 
	\hline 
	$S(t)$ & Sueldos asegurados totales en el año $t$. \\ 
	\hline 
	$CR(t)$ & Tasa de contribución en el año $t$. \\ 
	\hline 
	$i(t)$ & Tasa de interés en el año $t$. \\ 
	\hline 
\end{tabular} 	
\end{center}

De donde tienen las siguientes identidades:\\

El ingreso total anual al año $t$:
\[ R(t)=C(t)+I(t) .\]

El ingreso anual de ganancias generadas por interés al año $t$:

\[ I(t)=\left[ \sqrt{1+i(t)}-1\right] *\left[ C(t)-B(t)\right] +i(t)*V(t-1) . \]

El aumento anual en la reserva $V(t)$ al año $t$:
\[ \varDelta V(t)=V(t)-V(t-1)=R(t)-B(t) .\]


El ingreso de contribuciones anual al año $t$:
\[ C(t)=CR(t)*S(t) .\]


De donde para calcular el valor de la reserva $V(t)$ a un año $t$ se utiliza la fórmula recursiva:

\[ V(t)=\left[ 1+i(t)\right] *V(t-1)+\sqrt{1+i(t)}*\left[ CR(t)*S(t)-B(t)\right] , \]

por otro lado sea $v(t)=(1+i(t))^{-1}$ el factor de descuento asociado a la tasa de interés $i(t)$, entonces la ecuación anterior puede reescribirse como:

\[ v(t)*V(t)=V(t-1)+v(t)^{\frac{1}{2}}*\left[ CR(t)*S(t)-B(t)\right] . \]

De donde se calcula de forma recursiva el valor de la reserva $V(t)$, para determinar el período de equilibrio, es decir el período para el cual la reserva $V(t=0)\geq0$.\\

Para el cálculo de la $CR(n,m)$ para un intervalo de $(n,m)$ años, se hace uso de las siguientes:

\[ U(t)V(t)=U(t-1)*V(t-1)+CR(t)*\left[\overline{S(t)}-\overline{S(t-1)}\right]-\left[\overline{B(t)}-\overline{B(t-1)} \right]   ,\]

en donde:
\[ \overline{S(t)}=\sum\limits_{k=1}^{t}S(k)*W(k), \]
\[ \overline{B(t)}=\sum\limits_{k=1}^{t}B(k*W(k)) ,\]
\[ U(t)=\prod\limits_{k=1}^{t}v(k), \]
\[ W(t)=U(t-1)*v(t)^{\frac{1}{2}} \].

De donde la $CR^{intervalo}_{(n,m)}$ para un intervalo de tiempo $(n,m)$ sería:

\[CR^{intervalo}_{(n,m) }=\dfrac{\overline{B(m)}-\overline{B(n-1)}-V(n-1)}{\overline{S(m)}-\overline{S(n-1)}}\]

En el caso de la PE, se espera que cuando el régimen se encuentra en un período de madurez y la población asegurada y pensionada es estable, la cuota de contribución permanece constante indefinidamente.

\paragraph{Nota}Modelo extraído de (\cite{amssp}, pp. 73-78).

%%%%%%%%%%%%%%%%%%%%%%%%%%%%%%%%%%%%%%%%%%%%%%%%%%%%%%%%%%%%%%%%%%%%%%%%%%%%%%%%%%%%%%%%%%%%%%%%%%%%%%%%%%%%%%%%%%%%%%%%%%%%%%%%%%%%%%%%%%%%%%%%%%

\newpage

\bibliographystyle{apacite}
\bibliography{biblioigss}
\label{fin}


\par}

\end{document}